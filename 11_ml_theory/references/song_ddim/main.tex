
\documentclass{article} %
\usepackage{iclr2021_conference,times}
\usepackage{subcaption}
\usepackage{graphicx}
\input{math_commands.tex}
\usepackage{algorithm}
\usepackage{algpseudocode}
\usepackage{multirow}
\usepackage{array}
\usepackage{caption}
\usepackage{hyperref}
\hypersetup{colorlinks = true, linkcolor = brown,
            urlcolor  = gray,
            citecolor = brown,
            anchorcolor = brown}
\usepackage{xcolor}
\usepackage{colortbl}
\usepackage{tipa}
\allowdisplaybreaks
\title{Denoising Diffusion Implicit Models}

\iclrfinalcopy

\author{Jiaming Song, Chenlin Meng \& Stefano Ermon\\
Stanford University\\
\texttt{\{tsong,chenlin,ermon\}@cs.stanford.edu}}


\newcommand{\fix}{\marginpar{FIX}}
\newcommand{\new}{\marginpar{NEW}}

\begin{document}


\maketitle

\begin{abstract}
Denoising diffusion probabilistic models (DDPMs) have achieved high quality image generation without adversarial training, yet they require simulating a Markov chain for many steps in order to produce a sample. To accelerate sampling, we present denoising diffusion implicit models (DDIMs), a more efficient class of iterative implicit probabilistic models with the same training procedure as DDPMs. In DDPMs, the generative process is defined as the reverse of a particular Markovian diffusion process. We generalize DDPMs via a class of non-Markovian diffusion processes that lead to the same training objective. These non-Markovian processes can correspond to generative processes that are deterministic, giving rise to implicit models that produce high quality samples much faster. We empirically demonstrate that DDIMs can produce high quality samples $10 \times$ to $50 \times$ faster in terms of wall-clock time compared to DDPMs, allow us to trade off computation for sample quality, perform semantically meaningful image interpolation directly in the latent space, and reconstruct observations with very low error. 
\end{abstract}

\section{Introduction}
Deep generative models have demonstrated the ability to produce high quality samples in many domains~\citep{karras2020analyzing,oord2016wavenet}. In terms of image generation, generative adversarial networks (GANs, \citet{goodfellow2014generative}) currently exhibits higher sample quality than likelihood-based methods such as variational autoencoders~\citep{kingma2013auto}, autoregressive models~\citep{oord2016pixel} and normalizing flows~\citep{rezende2015variational,dinh2016density}. However, GANs require very specific choices in optimization and architectures in order to stabilize training~\citep{arjovsky2017wasserstein,gulrajani2017improved,karras2018a,brock2018large}, and could fail to cover modes of the data distribution~\citep{zhao2018bias}.

Recent works on iterative generative models~\citep{bengio2014deep}, such as denoising diffusion probabilistic models (DDPM, \citet{ho2020denoising}) and noise conditional score networks (NCSN, \citet{song2019generative}) %
have demonstrated the ability to produce samples comparable to that of GANs, without having to perform adversarial training. To achieve this, many denoising autoencoding models are trained to denoise samples corrupted by various levels of Gaussian noise. Samples are then produced by a Markov chain which, starting from white noise, progressively denoises it into an image. This generative Markov Chain process is either based on Langevin dynamics~\citep{song2019generative} or obtained by reversing a forward \textit{diffusion process} that progressively turns an image into noise~\citep{sohl-dickstein2015deep}. 

A critical drawback of these models is that they require many iterations to produce a high quality sample. For DDPMs, this is because that the generative process (from noise to data) approximates the reverse of the forward \textit{diffusion process} (from data to noise), which could have thousands of steps; iterating over all the steps is required to produce a single sample, which is much slower compared to GANs, which only needs one pass through a network. 
For example, it takes around 20 hours to sample 50k images of size $32 \times 32$ from a DDPM, but less than a minute to do so from \href{https://github.com/ajbrock/BigGAN-PyTorch}{\underline{a GAN}} on a Nvidia 2080 Ti GPU. This becomes more problematic for larger images as sampling 50k images of size $256 \times 256$ could take nearly $1000$ hours on the same GPU.

To close this efficiency gap between DDPMs and GANs, %
we present denoising diffusion implicit models (DDIMs). DDIMs are implicit probabilistic models~\citep{mohamed2016learning} and are closely related to DDPMs, in the sense that they are trained with the same objective function. %
In Section~\ref{sec:variational}, 
we generalize the forward \textit{diffusion process} used by DDPMs, which is Markovian, to \textit{non-Markovian} ones, for which we are still able to design suitable reverse generative Markov chains.
We show that the resulting variational training objectives have a shared surrogate objective, which is \textit{exactly} the objective used to train DDPM. %
Therefore, we can freely choose from a large family of generative models using the same neural network simply by choosing a different, \textit{non-Markovian} diffusion process (Section \ref{sec:reverse-family}) and the corresponding reverse generative Markov Chain. 
In particular, we are able to use \textit{non-Markovian} diffusion processes which lead to "short" generative Markov chains (Section \ref{sec:acceleration}) that can be simulated in a small number of steps.
This can massively increase sample efficiency only at a minor cost in sample quality. 




In Section~\ref{sec:experiments}, we demonstrate several empirical benefits of DDIMs over DDPMs. \textit{First}, DDIMs have superior sample generation quality compared to DDPMs, when we accelerate sampling by $10\times$ to $100\times$ using our proposed method. 
\textit{Second}, DDIM samples have the following ``consistency'' property, which does not hold for DDPMs: if we start with the same initial latent variable and generate several samples with Markov chains of various lengths, these samples would have similar high-level features. %
\textit{Third}, because of ``consistency'' in DDIMs, %
we can perform semantically meaningful image interpolation by manipulating the initial latent variable in DDIMs, unlike DDPMs which interpolates near the image space due to the stochastic generative process.



\section{Background}
\label{sec:background}
\begin{figure}
    \centering
    \begin{subfigure}{0.45\textwidth}
    \includegraphics[width=\textwidth]{figures/diffusion-orig.pdf}
    \end{subfigure}
    ~
    \begin{subfigure}{0.45\textwidth}
    \includegraphics[width=\textwidth]{figures/diffusion-generalized.pdf}
    \end{subfigure}
    \caption{Graphical models for diffusion (left) and non-Markovian (right) inference models.}
    \label{fig:diffusion}
\end{figure}
Given samples from a data distribution $q(\vx_0)$, we are interested in learning a model distribution $p_\theta(\vx_0)$ that approximates $q(\vx_0)$ and is easy to sample from. 
Denoising diffusion probabilistic models (DDPMs,~\citet{sohl-dickstein2015deep,ho2020denoising}) are latent variable models of the form
\begin{align}
    p_\theta(\vx_0) = \int p_\theta(\vx_{0:T}) \diff \vx_{1:T}, \quad \text{where} \quad  p_\theta(\vx_{0:T}) := p_\theta(\vx_T) \prod_{t=1}^{T} p^{(t)}_\theta(\vx_{t-1} | \vx_t) \label{eq:gen}
\end{align}
where $\vx_1, \ldots, \vx_T$ are latent variables in the same sample space as $\vx_0$ (denoted as $\gX$). The parameters $\theta$ are learned to fit the data distribution $q(\vx_0)$ by maximizing a variational lower bound: %
\begin{align}
    \max_\theta \bb{E}_{q(\vx_0)}[\log p_\theta(\vx_0)] \leq \max_\theta \bb{E}_{q(\vx_0, \vx_1, \ldots, \vx_T)}\left[\log p_\theta(\vx_{0:T}) - \log q(\vx_{1:T} | \vx_0) \right] \label{eq:elbo}
\end{align}
where $q(\vx_{1:T} | \vx_0)$ is some inference distribution over the latent variables.
Unlike typical latent variable models (such as the variational autoencoder~\citep{rezende2014stochastic}), DDPMs are learned with a fixed (rather than trainable) inference procedure $q(\vx_{1:T} | \vx_0)$, 
and latent variables are relatively high dimensional. 
For example, \citet{ho2020denoising} considered the following Markov chain with Gaussian transitions 
parameterized by a decreasing sequence
$\alpha_{1:T} \in (0, 1]^T$:
\begin{align}
    q(\vx_{1:T} | \vx_0) := \prod_{t=1}^{T} q(\vx_t | \vx_{t-1}), \text{where} \ q(\vx_t | \vx_{t-1}) := \gN\left(\sqrt{\frac{\alpha_t}{\alpha_{t-1}}} \vx_{t-1}, \left(1 - \frac{\alpha_t}{\alpha_{t-1}}\right) \mI\right) \label{eq:diff-ho}
\end{align}
where the covariance matrix is ensured to have positive terms on its diagonal.
This is called the \textit{forward process} due to the autoregressive nature of the sampling procedure (from $\vx_0$ to $\vx_T$). We call the latent variable model $p_\theta(\vx_{0:T})$, which is a Markov chain that samples from $\vx_T$ to $\vx_0$, the \textit{generative process}, since it approximates the intractable \textit{reverse process} $q(\vx_{t-1} | \vx_t)$. Intuitively, the forward process progressively adds noise to the observation $\vx_0$, whereas the generative process progressively denoises a noisy observation (\Figref{fig:diffusion}, left).

A special property of the forward process is that 
$$
q(\vx_t | \vx_0) := \int q(\vx_{1:t} | \vx_0) \diff \vx_{1:(t-1)}= \gN(\vx_t; \sqrt{\alpha_t} \vx_0, (1 - \alpha_t) \mI);
$$ 
so we can express $\vx_t$ as a linear combination of $\vx_0$ and a noise variable $\epsilon$:
\begin{align}
    \vx_t = \sqrt{\alpha_t} \vx_0 + \sqrt{1 - \alpha_t} \epsilon, \quad \text{where} \quad \epsilon \sim \gN(\vzero, \mI). \label{eq:reparam_xt}
\end{align}
When we set $\alpha_{T}$ sufficiently close to $0$, $q(\vx_T | \vx_0)$ converges to a standard Gaussian for all $\vx_0$, so it is natural to set $p_\theta(\vx_T) := \gN(\vzero, \mI)$.
If all the conditionals %
are modeled as Gaussians with trainable mean functions and fixed variances, the objective in \eqref{eq:elbo} can be simplified to\footnote{Please refer to Appendix~\ref{app:ddpm} for details.}:
\begin{align}
    L_\gamma(\epsilon_\theta) & := \sum_{t=1}^{T}\gamma_t\bb{E}_{\vx_0 \sim q(\vx_0), \epsilon_t \sim \gN(\vzero, \mI)}\left[  \norm{\epsilon_{\theta}^{(t)}(\sqrt{\alpha_t} \vx_0 + \sqrt{1 - \alpha_t} \epsilon_t) - \epsilon_t}_2^2 \right] \label{eq:l-gamma}
\end{align}
where $\epsilon_\theta := \{\epsilon_\theta^{(t)}\}_{t=1}^{T}$ is a set of $T$ functions, each $\epsilon_{\theta}^{(t)}: \gX \to \gX$ (indexed by $t$) is a function with trainable parameters $\theta^{(t)}$, and $\gamma := [\gamma_1, \ldots, \gamma_T]$ is a vector of positive coefficients in the objective that depends on $\alpha_{1:T}$. 
In \citet{ho2020denoising}, the objective with $\gamma = \vone$ is optimized instead to maximize generation performance of the trained model; this is also the same objective used in noise conditional score networks~\citep{song2019generative} based on score matching~\citep{h2005estimation,vincent2011connection}. %
From a trained model, $\vx_0$ is sampled by first sampling $\vx_T$ from the prior $p_\theta(\vx_T)$, and then sampling $\vx_{t-1}$ from the generative processes iteratively. %


The length $T$ of the forward process is an important hyperparameter in DDPMs.
From a variational perspective, a large $T$ allows the reverse process to be close to a Gaussian~\citep{sohl-dickstein2015deep}, so that the generative process modeled with Gaussian conditional distributions becomes a good approximation; this motivates the choice of large $T$ values, such as $T = 1000$ in \citet{ho2020denoising}. %
However, as all $T$ iterations have to be performed sequentially, instead of in parallel, to obtain a sample $\vx_0$, sampling from DDPMs is much slower than sampling from other deep generative models, which makes them impractical for tasks where compute is limited and latency is critical. %

\input{sections/3_method}
\section{Experiments}
\label{sec:experiments}

In this section, we show that DDIMs outperform DDPMs in terms of image generation when fewer iterations are considered, giving speed ups of $10\times$ to $100\times$ over the original DDPM generation process. %
Moreover, unlike DDPMs, once the initial latent variables $\vx_T$ are fixed, DDIMs retain high-level image features regardless of the generation trajectory, so they are able to perform interpolation directly from the latent space. DDIMs can also be used to encode samples that reconstruct them from the latent code, which DDPMs cannot do due to the stochastic sampling process.

For each dataset, we use the \textbf{same trained model} with $T = 1000$ and the objective being $L_\gamma$ from \eqref{eq:l-gamma} with $\gamma = \vone$; as we argued in Section~\ref{sec:variational}, no changes are needed with regards to the training procedure.
The only changes that we make is \textbf{how we produce samples from the model}; we achieve this by controlling $\tau$ (which controls how fast the samples are obtained) and $\sigma$ (which interpolates between the deterministic DDIM and the stochastic DDPM).

We consider different sub-sequences $\tau$ of $[1, \ldots, T]$ and different variance hyperparameters $\sigma$ indexed by elements of $\tau$. To simplify comparisons, we consider $\sigma$ with the form:
\begin{align}
    \sigma_{\tau_{i}}(\eta) = \eta \sqrt{(1 - \alpha_{\tau_{i-1}})/(1 - \alpha_{\tau_{i}})}\sqrt{1 - {\alpha_{\tau_i}}/{\alpha_{\tau_{i-1}}}}, %
\end{align}
where $\eta \in \R_{\geq 0}$ is a hyperparameter that we can directly control. This includes an original DDPM generative process when $\eta = 1$ and DDIM when $\eta = 0$. We also consider DDPM where the random noise has a larger standard deviation than $\sigma(1)$, which we denote as $\hat{\sigma}$:
$
    \hat{\sigma}_{\tau_{i}} = \sqrt{1 - {\alpha_{\tau_i}}/{\alpha_{\tau_{i-1}}}}%
$
. This is used by \href{https://github.com/hojonathanho/diffusion/blob/master/scripts/run_cifar.py#L136}{\underline{the implementation}} in \cite{ho2020denoising} \textbf{only to obtain the CIFAR10 samples}, but not samples of the other datasets. We include more details in Appendix~\ref{app:exp}. %

\subsection{Sample quality and efficiency}
In Table~\ref{tab:cifar10-fid}, we report the quality of the generated samples with models trained on CIFAR10 and CelebA, as measured by Frechet  Inception Distance (FID~\citep{heusel2017gans}), where we vary the number of timesteps used to generate a sample ($\dim(\tau)$) and the stochasticity of the process ($\eta$).
As expected, the sample quality becomes higher as we increase $\dim(\tau)$, presenting a trade-off between sample quality and computational costs.
We observe that DDIM ($\eta = 0$) achieves the best sample quality when $\dim(\tau)$ is small, and DDPM ($\eta = 1$ and $\hat{\sigma}$) typically has worse sample quality compared to its less stochastic counterparts with the same $\dim(\tau)$, except for the case for $\dim(\tau) = 1000$ and $\hat{\sigma}$ reported by \citet{ho2020denoising} where DDIM is marginally worse. However, the sample quality of $\hat{\sigma}$ becomes much worse for smaller $\dim(\tau)$, which suggests that it is ill-suited for shorter trajectories. DDIM, on the other hand, achieves high sample quality much more consistently. 

In \Figref{fig:cifar10-tvh}, we show CIFAR10 and CelebA samples with the same number of sampling steps and varying $\sigma$. For the DDPM, the sample quality deteriorates rapidly when the sampling trajectory has 10 steps. For the case of $\hat{\sigma}$, the generated images seem to have more noisy perturbations under short trajectories; this explains why the FID scores are much worse than other methods, as FID is very sensitive to such perturbations (as discussed in \citet{Jolicoeur-Martineau2020adversarial}). 

In \Figref{fig:time}, we show that the amount of time needed to produce a sample scales linearly with the length of the sample trajectory. %
This suggests that DDIM is useful for producing samples more efficiently, as samples can be generated in much fewer steps. Notably, DDIM is able to produce samples with quality comparable to 1000 step models within $20$ to $100$ steps, which is a $10\times$ to $50\times$ speed up compared to the original DDPM. Even though DDPM could also achieve reasonable sample quality with $100\times$ steps, DDIM requires much fewer steps to achieve this; on CelebA, the FID score of the 100 step DDPM is similar to that of the 20 step DDIM. 

\newcolumntype{G}{>{\centering}b{0.065\textwidth}}
\newcolumntype{C}{>{\centering\arraybackslash}b{0.065\textwidth}}
\definecolor{LightCyan}{rgb}{0.88,1,1}
\definecolor{LightOrange}{rgb}{1,0.85,0.70}
\definecolor{DarkCyan}{rgb}{0,0.8,0.8}
\definecolor{DarkOrange}{rgb}{1,0.50,0.0}
\begin{table}
    \centering
    \caption{CIFAR10 and CelebA image generation measured in FID. $\eta = 1.0$ and $\hat{\sigma}$ are cases of {\color{DarkOrange}DDPM} (although \citet{ho2020denoising} only considered $T=1000$ steps, and $S < T$ can be seen as simulating DDPMs trained with $S$ steps), and $\eta = 0.0$ indicates {\color{DarkCyan}DDIM}. %
    }
    \resizebox{\textwidth}{!}{
    \begin{tabular}{lc|GGGGG|GGGGC}
    \toprule
    &  & \multicolumn{5}{c|}{CIFAR10 ($32 \times 32$)} & \multicolumn{5}{c}{CelebA ($64 \times 64$)} \\
    \multicolumn{2}{c|}{$S$} & 10 & 20 & 50 & 100 & 1000 & 10 & 20 & 50 & 100 & 1000 \\\midrule
    \rowcolor{LightCyan}
    \multirow{4}{*}{$\eta$} & $0.0$ & \textbf{13.36} & \textbf{6.84} & \textbf{4.67} & \textbf{4.16} & 4.04 & \textbf{17.33} & \textbf{13.73} & \textbf{9.17} & \textbf{6.53} & 3.51 \\
    & 0.2 & 14.04 & 7.11 & 4.77 & 4.25 & 4.09 & 17.66 & 14.11 & 9.51 & 6.79 & 3.64 \\
    & 0.5 & 16.66 & 8.35 & 5.25 & 4.46 & 4.29 & 19.86 & 16.06 & 11.01 & 8.09 & 4.28 \\
    \rowcolor{LightOrange}
    & 1.0 & 41.07 & 18.36 & 8.01 & 5.78 & 4.73  & 33.12 & 26.03 & 18.48 & 13.93 & 5.98 \\\midrule
    \rowcolor{LightOrange}
    \multicolumn{2}{c|}{$\hat{\sigma}$} & 367.43 & 133.37 & 32.72 & 9.99 & \textbf{3.17} & 299.71 & 183.83 & 71.71 & 45.20 & \textbf{3.26} \\
        \bottomrule
    \end{tabular}
    }
    \label{tab:cifar10-fid}
\end{table}

\begin{figure}
    \centering
    \begin{subfigure}{0.23\textwidth}
    \includegraphics[width=\textwidth]{figures/cifar10-dim10.pdf}
    \end{subfigure}
    ~
    \begin{subfigure}{0.23\textwidth}
    \includegraphics[width=\textwidth]{figures/cifar10-dim100.pdf}
    \end{subfigure}
    ~
        \begin{subfigure}{0.23\textwidth}
    \includegraphics[width=\textwidth]{figures/celeba-dim10.pdf}
    \end{subfigure}
    ~
    \begin{subfigure}{0.23\textwidth}
    \includegraphics[width=\textwidth]{figures/celeba-dim100.pdf}
    \end{subfigure}
    \caption{CIFAR10 and CelebA samples with $\dim(\tau) = 10$ and $\dim(\tau) = 100$.}
    \label{fig:cifar10-tvh}
\end{figure}

\begin{figure}
    \centering
    \begin{subfigure}{0.4\textwidth}
    \centering
        \includegraphics[width=\textwidth]{figures/cifar10-tvh.pdf}
    \end{subfigure}
    ~
     \begin{subfigure}{0.4\textwidth}
     \centering
        \includegraphics[width=\textwidth]{figures/bedroom-tvh.pdf}
    \end{subfigure}
    \caption{Hours to sample 50k images with one Nvidia 2080 Ti GPU and samples at different steps.}
    \label{fig:time}
\end{figure}

\subsection{Sample consistency in DDIMs} 
For DDIM, the generative process is deterministic, and $\vx_0$ would depend only on the initial state $\vx_T$. In \Figref{fig:consistency}, we observe the generated images under different generative trajectories (i.e. different $\tau$) while starting with the same initial $\vx_T$. Interestingly, for the generated images with the same initial $\vx_T$, most high-level features are similar, regardless of the generative trajectory. In many cases, samples generated with only 20 steps are already very similar to ones generated with 1000 steps in terms of high-level features, with only minor differences in details. 
Therefore, it would appear that $\vx_T$ alone would be an informative latent encoding of the image; and minor details that affects sample quality are encoded in the parameters, as longer sample trajectories gives better quality samples but do not significantly affect the high-level features.
We show more samples in Appendix~\ref{app:samples}. %

\begin{figure}
\begin{subfigure}{\textwidth}
    \centering
    \includegraphics[width=0.95\textwidth]{figures/cifar10-consistency.pdf}
\end{subfigure}

\begin{subfigure}{\textwidth}
    \centering
    \includegraphics[width=0.99\textwidth]{figures/bedroom-consistency.pdf}
\end{subfigure}

\begin{subfigure}{\textwidth}
    \centering
    \includegraphics[width=0.99\textwidth]{figures/church-consistency.pdf}
\end{subfigure}
\caption{Samples from DDIM with the same random $\vx_T$ and different number of steps.}
\label{fig:consistency}
\end{figure}

\subsection{Interpolation in deterministic generative processes} 
\begin{figure}[h]
\begin{subfigure}{\textwidth}
    \centering
    \includegraphics[width=\textwidth]{figures/celeba-interp-line.png}
\end{subfigure}
~
\begin{subfigure}{\textwidth}
    \centering
    \includegraphics[width=\textwidth]{figures/bedroom-interp-line.png}
\end{subfigure}
~
\begin{subfigure}{\textwidth}
    \centering
    \includegraphics[width=\textwidth]{figures/church-interp-line.png}
\end{subfigure}
\caption{Interpolation of samples from DDIM with $\dim(\tau) = 50$.}
\label{fig:interp-line}
\end{figure}
Since the high level features of the DDIM sample is encoded by $\vx_T$, we are interested to see whether it would exhibit the semantic interpolation effect similar to that observed in other implicit probabilistic models, such as GANs~\citep{goodfellow2014generative}. This is different from the interpolation procedure in~\citet{ho2020denoising}, since in DDPM the same $\vx_T$ would lead to highly diverse $\vx_0$ due to the stochastic generative process\footnote{Although it might be possible if one interpolates all $T$ noises, like what is done in \citet{song2020improved}.}. In \Figref{fig:interp-line}, we show that simple interpolations in $\vx_T$ can lead to semantically meaningful interpolations between two samples. We include more details and samples in Appendix~\ref{app:interpolation}. This allows DDIM to control the generated images on a high level directly through the latent variables, which DDPMs cannot.

\subsection{Reconstruction from Latent Space}
As DDIM is the Euler integration for a particular ODE, it would be interesting to see whether it can encode from $\vx_0$ to $\vx_T$ (reverse of \eqref{eq:ddim-ode}) and reconstruct $\vx_0$ from the resulting $\vx_T$ (forward of \eqref{eq:ddim-ode})\footnote{Since $\vx_T$ and $\vx_0$ have the same dimensions, their compression qualities are not our immediate concern.}. We consider encoding and decoding on the CIFAR-10 test set with the CIFAR-10 model with $S$ steps for both encoding and decoding; we report the per-dimension mean squared error (scaled to $[0, 1]$) in Table~\ref{tab:cifar10-recon}. Our results show that DDIMs have lower reconstruction error for larger $S$ values and have properties similar to Neural ODEs and normalizing flows.
The same cannot be said for DDPMs due to their stochastic nature.
\begin{table}
    \centering
    \caption{Reconstruction error with DDIM on CIFAR-10 test set, rounded to $10^{-4}$.}
    \begin{tabular}{l|ccccccc}
    \toprule
    $S$ & 10 & 20 & 50 & 100 & 200 & 500 & 1000 \\
    \midrule
    Error & 0.014 & 0.0065 & 0.0023 & 0.0009 & 0.0004 & 0.0001 & $0.0001$ \\
    \bottomrule
    \end{tabular}
    \label{tab:cifar10-recon}
\end{table}
\section{Related Work}
Our work is based on a large family of existing methods on learning generative models as transition operators of Markov chains~\citep{sohl-dickstein2015deep,bengio2014deep,salimans2014markov,song2017a,goyal2017variational,levy2017generalizing}. Among them, denoising diffusion probabilistic models (DDPMs, \citet{ho2020denoising}) and noise conditional score networks (NCSN, \citet{song2019generative,song2020improved}) have recently achieved high sample quality comparable to GANs~\citep{brock2018large,karras2018a}. DDPMs optimize a variational lower bound to the log-likelihood, whereas NCSNs optimize the score matching objective~\citep{h2005estimation} over a nonparametric Parzen density estimator of the data~\citep{vincent2011connection,raphan2011least}. 

Despite their different motivations, DDPMs and NCSNs are closely related. Both use a denoising autoencoder objective for many noise levels, and both use a procedure similar to Langevin dynamics to produce samples~\citep{neal2011mcmc}. Since Langevin dynamics is a discretization of a gradient flow~\citep{jordan1998variational}, %
both DDPM and NCSN require many steps to achieve good sample quality. %
This aligns with the observation that DDPM and existing NCSN methods have trouble generating high-quality samples in a few iterations.

DDIM, on the other hand, is an implicit generative model~\citep{mohamed2016learning} where samples are uniquely determined from the latent variables. Hence, DDIM has certain properties that resemble GANs~\citep{goodfellow2014generative} and invertible flows~\citep{dinh2016density}, such as the ability to produce semantically meaningful interpolations. We derive DDIM from a purely variational perspective, where the restrictions of Langevin dynamics are not relevant; this could partially explain why we are able to observe superior sample quality compared to DDPM under fewer iterations. %
The sampling procedure of DDIM is also reminiscent of neural networks with continuous depth~\citep{chen2018neural,grathwohl2018ffjord}, since the samples it produces from the same latent variable have similar high-level visual features, regardless of the specific sample trajectory.



\section{Discussion}
We have presented DDIMs -- an implicit generative model trained with denoising auto-encoding / score matching objectives -- from a purely variational perspective. DDIM is able to generate high-quality samples much more efficiently than existing DDPMs and NCSNs, with the ability to perform meaningful interpolations from the latent space. The non-Markovian forward process presented here seems to suggest  continuous forward processes other than Gaussian (which cannot be done in the original diffusion framework, since Gaussian is the only stable distribution with finite variance). We also demonstrated a discrete case with a multinomial forward process in Appendix~\ref{app:discrete}, and it would be interesting to investigate similar alternatives for other combinatorial structures.

Moreover, since the sampling procedure of DDIMs is similar to that of an neural ODE, it would be interesting to see if methods that decrease the discretization error in ODEs, including multi-step methods such as Adams-Bashforth~\citep{butcher2008numerical}, could be helpful for further improving sample quality in fewer steps~\citep{queiruga2020continuous}. It is also relevant to investigate whether DDIMs exhibit other properties of existing implicit models~\citep{bau2019seeing}.

\section*{Acknowledgements}
The authors would like to thank Yang Song and Shengjia Zhao for helpful discussions over the ideas, Kuno Kim for reviewing an earlier draft of the paper, and Sharvil Nanavati and Sophie Liu for identifying typos. This research was supported by NSF (\#1651565, \#1522054, \#1733686), ONR (N00014-19-1-2145), AFOSR (FA9550-19-1-0024), and Amazon AWS.


\bibliography{iclr2021_conference}
\bibliographystyle{iclr2021_conference}

\newpage
\appendix

\section{Non-Markovian Forward Processes for a Discrete Case}
\label{app:discrete}
In this section, we describe a non-Markovian forward processes for discrete data and corresponding variational objectives. Since the focus of this paper is to accelerate reverse models corresponding to the Gaussian diffusion, we leave empirical evaluations as future work. 

For a categorical observation $\vx_0$ that is a one-hot vector with $K$ possible values, we define the forward process as follows. First, we have $q(\vx_t | \vx_0)$ as the following categorical distribution:
\begin{align}
    q(\vx_t | \vx_0) = \mathrm{Cat}(\alpha_t \vx_0 + (1 - \alpha_t) \vone_K)
\end{align}
where $\vone_K \in \R^{K}$ is a vector with all entries being $1/K$, and $\alpha_t$ decreasing from $\alpha_0 = 1$ for $t = 0$ to $\alpha_T = 0$ for $t = T$.
Then we define $q(\vx_{t-1} | \vx_t, \vx_0)$ as the following mixture distribution:
\begin{align}
    q(\vx_{t-1} | \vx_t, \vx_0) = \begin{cases}
    \mathrm{Cat}(\vx_t) & \text{with probability } \sigma_t \\
    \mathrm{Cat}(\vx_0) & \text{with probability } (\alpha_{t-1} - \sigma_t \alpha_t) \\
    \mathrm{Cat}(\vone_K) & \text{with probability } (1 - \alpha_{t-1}) - (1 - \alpha_t) \sigma_t
    \end{cases},
\end{align}
or equivalently:
\begin{align}
    q(\vx_{t-1} | \vx_t, \vx_0) = \mathrm{Cat}\left(\sigma_t \vx_t + (\alpha_{t-1} - \sigma_t \alpha_t) \vx_0 + ((1 - \alpha_{t-1}) - (1 - \alpha_t) \sigma_t) \vone_K\right),
\end{align}
which is consistent with how we have defined $q(\vx_t | \vx_0)$. 

Similarly, we can define our reverse process $p_\theta(\vx_{t-1} | \vx_t)$ as:
\begin{align}
    p_\theta(\vx_{t-1} | \vx_t) = \mathrm{Cat}\left(\sigma_t \vx_t + (\alpha_{t-1} - \sigma_t \alpha_t) f_\theta^{(t)}(\vx_t) + ((1 - \alpha_{t-1}) - (1 - \alpha_t) \sigma_t) \vone_K\right),
\end{align}
where $f_\theta^{(t)}(\vx_t)$ maps $\vx_t$ to a $K$-dimensional vector. As $(1 - \alpha_{t-1}) - (1 - \alpha_t) \sigma_t \to 0$, the sampling process will become less stochastic, in the sense that it will either choose $\vx_t$ or the predicted $\vx_0$ with high probability.
The KL divergence
\begin{align}
    \KL(q(\vx_{t-1} | \vx_t, \vx_0) \Vert p_\theta(\vx_{t-1} | \vx_t))
\end{align}
is well-defined, and is simply the KL divergence between two categoricals. Therefore, the resulting variational objective function should be easy to optimize as well. Moreover, as KL divergence is convex, we have this upper bound (which is tight when the right hand side goes to zero):
\begin{align}
    \KL(q(\vx_{t-1} | \vx_t, \vx_0) \Vert p_\theta(\vx_{t-1} | \vx_t)) \leq (\alpha_{t-1} - \sigma_t \alpha_t) \KL(\mathrm{Cat}(\vx_0) \Vert \mathrm{Cat}(f_\theta^{(t)}(\vx_t))). \nonumber
\end{align}
The right hand side is simply a multi-class classification loss (up to constants), so we can arrive at similar arguments regarding how changes in $\sigma_t$ do not affect the objective (up to re-weighting).

\section{Proofs}
\label{app:proofs}

\begin{lemma}
\label{lemma:reverse-process-consistency}
For $q_\sigma(\vx_{1:T} | \vx_0)$ defined in \eqref{eq:diff-new} and $q_\sigma(\vx_{t-1} | \vx_t, \vx_0)$ defined in \eqref{eq:reversed-close-form}, we have:
\begin{align}
q_\sigma(\vx_{t} | \vx_0) = \gN(\sqrt{\alpha_t} \vx_0, (1 - \alpha_t) \mI)
\end{align}
\end{lemma}
\begin{proof}
Assume for any $t \leq T$, $q_\sigma(\vx_{t} | \vx_0) = \gN(\sqrt{\alpha_t} \vx_0, (1 - \alpha_t) \mI)$ holds, if:
\begin{align}
    q_\sigma(\vx_{t-1} | \vx_0) = \gN(\sqrt{\alpha_{t-1}} \vx_0, (1 - \alpha_{t-1}) \mI)
\end{align}
then we can prove the statement with an induction argument for $t$ from $T$ to $1$, since the base case ($t = T$) already holds.

First, we have that
$$
    q_\sigma(\vx_{t-1} | \vx_0) := \int_{\vx_t} q_\sigma(\vx_{t} | \vx_0) q_\sigma(\vx_{t-1} | \vx_t, \vx_0) \diff \vx_t
$$
and
\begin{gather}
   q_\sigma(\vx_{t} | \vx_0) = \gN(\sqrt{\alpha_t} \vx_0, (1 - \alpha_t) \mI) \\
   q_\sigma(\vx_{t-1} | \vx_t, \vx_0) = \gN\left(\sqrt{\alpha_{t-1}} \vx_{0} + \sqrt{1 - \alpha_{t-1} - \sigma^2_t} \cdot {\frac{\vx_{t}  - \sqrt{\alpha_{t}} \vx_0}{\sqrt{1 - \alpha_{t}}}}, \sigma_t^2 \mI \right).
\end{gather}
From \citet{bishop2006pattern} (2.115), we have that $q_\sigma(\vx_{t-1} | \vx_0)$ is Gaussian, denoted as $\gN(\mu_{t-1}, \Sigma_{t-1})$ where
\begin{align}
    \mu_{t-1} & = \sqrt{\alpha_{t-1}} \vx_{0} + \sqrt{1 - \alpha_{t-1} - \sigma^2_t} \cdot {\frac{\sqrt{\alpha_t} \vx_0  - \sqrt{\alpha_{t}} \vx_0}{\sqrt{1 - \alpha_{t}}}} \\
    & = \sqrt{\alpha_{t-1}} \vx_{0}
\end{align}
and
\begin{align}
    \Sigma_{t-1} & = \sigma_t^2 \mI + \frac{1 - \alpha_{t-1} - \sigma^2_t}{1 - \alpha_t} (1 - \alpha_t) \mI = (1 - \alpha_{t-1}) \mI
\end{align}
Therefore, $q_\sigma(\vx_{t-1} | \vx_0) = \gN(\sqrt{\alpha_{t-1}} \vx_0, (1 - \alpha_{t-1}) \mI)$, which allows us to apply the induction argument.
\end{proof}


\unifiedklobj*

\begin{proof}
From the definition of $J_\sigma$:
\begin{align}
   J_\sigma(\epsilon_\theta) & := \bb{E}_{\vx_{0:T} \sim q(\vx_{0:T})} \left[\log q_\sigma(\vx_T | \vx_0) + \sum_{t=2}^{T} \log q_\sigma(\vx_{t-1} | \vx_t, \vx_0) - \sum_{t=1}^{T} \log p_\theta^{(t)}(\vx_{t-1} | \vx_t) \right] \\
   & \equiv \bb{E}_{\vx_{0:T} \sim q(\vx_{0:T})} \left[\sum_{t=2}^{T} \KL(q_\sigma(\vx_{t-1} | \vx_t, \vx_0)) \Vert p_\theta^{(t)}(\vx_{t-1} | \vx_t)) - \log p_\theta^{(1)}(\vx_0 | \vx_1)  \right] \nonumber
\end{align}
where we use $\equiv$ to denote ``equal up to a value that does not depend on $\epsilon_\theta$ (but may depend on $q_\sigma$)''. For $t > 1$:
\begin{align}
   & \bb{E}_{\vx_{0}, \vx_t \sim q(\vx_{0}, \vx_t)} [\KL(q_\sigma(\vx_{t-1} | \vx_t, \vx_0)) \Vert p_\theta^{(t)}(\vx_{t-1} | \vx_t))] \nonumber \\
   = & \ \bb{E}_{\vx_{0}, \vx_t \sim q(\vx_{0}, \vx_t)}[\KL(q_\sigma(\vx_{t-1} | \vx_t, \vx_0)) \Vert q_\sigma(\vx_{t-1} | \vx_t, f_{\theta}^{(t)}(\vx_t)))] \nonumber \\
   \equiv & \ \bb{E}_{\vx_{0}, \vx_t \sim q(\vx_{0}, \vx_t)}\left[\frac{\norm{\vx_0 - f_{\theta}^{(t)}(\vx_t)}_2^2}{2 \sigma_t^2}\right] \\
   = & \ \bb{E}_{\vx_{0} \sim q(\vx_{0}), \epsilon \sim \gN(\vzero, \mI), \vx_t = \sqrt{\alpha_t} \vx_0 + \sqrt{1 - \alpha_t} \epsilon}\left[\frac{\norm{\frac{{(\vx_t - \sqrt{1 - \alpha_t} \epsilon)}}{\sqrt{\alpha_t}} - \frac{(\vx_t - \sqrt{1 - \alpha_t} \epsilon_{\theta}^{(t)}(\vx_t))}{\sqrt{\alpha_t}}}_2^2}{2 \sigma_t^2}\right] \\
   = & \ \bb{E}_{\vx_{0} \sim q(\vx_{0}), \epsilon \sim \gN(\vzero, \mI), \vx_t = \sqrt{\alpha_t} \vx_0 + \sqrt{1 - \alpha_t} \epsilon}\left[\frac{\norm{\epsilon - \epsilon_{\theta}^{(t)}(\vx_t)}_2^2}{2 d \sigma_t^2 \alpha_t}\right]
\end{align}
where $d$ is the dimension of $\vx_0$. For $t = 1$:
\begin{align}
   & \ \bb{E}_{\vx_{0}, \vx_1 \sim q(\vx_{0}, \vx_1)} \left[ - \log p_\theta^{(1)}(\vx_0 | \vx_1)  \right] \equiv \bb{E}_{\vx_{0}, \vx_1 \sim q(\vx_{0}, \vx_1)}\left[\frac{\norm{\vx_0 - f_{\theta}^{(t)}(\vx_1)}_2^2}{2 \sigma_1^2}\right] \\
   = & \ \bb{E}_{\vx_{0} \sim q(\vx_{0}), \epsilon \sim \gN(\vzero, \mI), \vx_1 = \sqrt{\alpha_1} \vx_0 + \sqrt{1 - \alpha_t} \epsilon}\left[\frac{\norm{\epsilon - \epsilon_{\theta}^{(1)}(\vx_1)}_2^2}{2 d \sigma_1^2 \alpha_1}\right]
\end{align}
Therefore, when $\gamma_t = 1 / (2 d \sigma_t^2 \alpha_t)$ for all $t \in \{1, \ldots, T\}$, we have
\begin{align}
    J_\sigma(\epsilon_\theta) \equiv \sum_{t=1}^{T} \frac{1}{2 d \sigma_t^2 \alpha_t} \bb{E}\left[\norm{\epsilon_{\theta}^{(t)}(\vx_t) - \epsilon_t}_2^2 \right] = L_\gamma(\epsilon_\theta)
\end{align}
for all $\epsilon_\theta$. From the definition of ``$\equiv$'', we have that $J_\sigma = L_\gamma + C$.
\end{proof}

\equivalence*

\begin{proof}
In the context of the proof, we consider $t$ as a continous, independent ``time'' variable and $\vx$ and $\alpha$ as functions of $t$. First, let us consider a reparametrization between DDIM and the VE-SDE\footnote{Refer to \citep{song2020score} for more details of VE-SDE.} by introducing the variables $\bar{\vx}$ and $\sigma$:
\begin{align}
    \bar{\vx}(t) = \bar{\vx}(0) + \sigma(t) \epsilon, \quad \epsilon \sim \gN(0, \mI),
\end{align}
for $t \in [0, \infty)$ and an increasing continuous function $\sigma: \bb{R}_{\geq 0} \to \bb{R}_{\geq 0}$ where $\sigma(0) = 0$. 

We can then define $\alpha(t)$ and $\vx(t)$ corresponding to DDIM case as:
\begin{gather}
    \bar{\vx}(t) = \frac{\vx(t)}{\sqrt{\alpha(t)}} \\
    \sigma(t) = \sqrt{\frac{1 - \alpha(t)}{\alpha(t)}}.
\end{gather}
This also means that:
\begin{gather}
    \vx(t) = \frac{\bar{\vx}(t)}{\sqrt{\sigma^2(t) + 1}} \\
    \alpha(t) = \frac{1}{1 + \sigma^2(t)},
\end{gather}
which establishes an bijection between $(\vx, \alpha)$ and $(\bar{\vx}, \sigma)$. From \Eqref{eq:reparam_xt} we have (note that $\alpha(0) = 1$):
\begin{align}
    \frac{\vx(t)}{\sqrt{\alpha(t)}} = \frac{\vx(0)}{\sqrt{\alpha(0)}} + \sqrt{\frac{1 - \alpha(t)}{\alpha(t)}} \epsilon, \quad \epsilon \sim \gN(0, \mI)
\end{align}
which can be reparametrized into a form that is consistent with VE-SDE:
\begin{align}
    \bar{\vx}(t) = \bar{\vx}(0) + \sigma(t) \epsilon.
\end{align}

Now, we derive the ODE forms for both DDIM and VE-SDE and show that they are equivalent.

\paragraph{ODE form for DDIM} We repeat \Eqref{eq:ddim-euler} here:
\begin{align}
    \frac{\vx_{t-\Delta t}}{\sqrt{\alpha_{t-\Delta t}}}  = \frac{\vx_t}{\sqrt{\alpha_t}}  + \left(\sqrt{\frac{1 - \alpha_{t-\Delta t}}{\alpha_{t-\Delta t}}} - \sqrt{\frac{1 - \alpha_{t}}{\alpha_t}}\right) \epsilon_\theta^{(t)}(\vx_t),
\end{align}
which is equivalent to:
\begin{align}
    \bar{\vx}(t-\Delta t) = \bar{\vx}(t) + (\sigma(t-\Delta t) - \sigma(t)) \cdot \epsilon_\theta^{(t)}(\vx(t))
\end{align}
Divide both sides by $(-\Delta t)$ and as $\Delta t \to 0$, we have:
\begin{align}
    \frac{\diff \bar{\vx}(t)}{\diff t} = \frac{\diff \sigma(t)}{\diff t} \epsilon_\theta^{(t)}\left(\frac{\bar{\vx}(t)}{\sqrt{\sigma^2(t) + 1}}\right), \label{eq:ddim:differential}
\end{align}
which is exactly what we have in \Eqref{eq:ddim-ode}. 

We note that for the optimal model, $\epsilon_\theta^{(t)}$ is a minimizer:
\begin{align}
    \epsilon_\theta^{(t)} = \argmin_{f_t} \bb{E}_{\vx(0) \sim q(\vx), \epsilon \sim \gN(0, \mI)}[\norm{f_t(\vx(t)) - \epsilon}_2^2] \label{eq:dsm-ddim}
\end{align}
where $\vx(t) = \sqrt{\alpha(t)} \vx(t) + \sqrt{1 - \alpha(t)} \epsilon$.

\paragraph{ODE form for VE-SDE} Define $p_t(\bar{\vx})$ as the data distribution perturbed with $\sigma^2(t)$ variance Gaussian noise. The probability flow for VE-SDE is defined as \cite{song2020score}:
\begin{align}
    \diff \bar{\vx} = -\frac{1}{2} g(t)^2 \nabla_{\bar{\vx}} \log p_t(\bar{\vx}) \diff t  \label{eq:yang-ode}
\end{align}
where $g(t) = \sqrt{\frac{\diff \sigma^2(t)}{\diff t}}$ is the diffusion coefficient, and $\nabla_{\bar{\vx}} \log p_t(\bar{\vx})$ is the score of $p_t$. 

The $\sigma(t)$-perturbed score function $\nabla_{\bar{\vx}} \log p_t(\bar{\vx})$ is also a minimizer (from denoising score matching~\citep{vincent2011connection}):
\begin{align}
    \nabla_{\bar{\vx}} \log p_t = \argmin_{g_t} \bb{E}_{\vx(0) \sim q(\vx), \epsilon \sim \gN(0, \mI)}[\norm{g_t(\bar{\vx}) + \epsilon / \sigma(t)}_2^2] \label{eq:dsm-ve}
\end{align}
where $\bar{\vx}(t) = \bar{\vx}(t) + \sigma(t) \epsilon$. 

Since there is an equivalence between $\vx(t)$ and $\bar{\vx}(t)$, we have the following relationship:
\begin{gather}
    \nabla_{\bar{\vx}} \log p_t(\bar{\vx}) = - \frac{\epsilon_\theta^{(t)}\left(\frac{\bar{\vx}(t)}{\sqrt{\sigma^2(t) + 1}}\right)}{\sigma(t)} \label{eq:score-equivalence}
\end{gather}
from \Eqref{eq:dsm-ddim} and \Eqref{eq:dsm-ve}. Plug \Eqref{eq:score-equivalence} and definition of $g(t)$ in \Eqref{eq:yang-ode}, we have:
\begin{align}
    \diff \bar{\vx}(t) = \frac{1}{2} \frac{\diff \sigma^2(t)}{\diff t} \frac{\epsilon_\theta^{(t)}\left(\frac{\bar{\vx}(t)}{\sqrt{\sigma^2(t) + 1}}\right)}{\sigma(t)} \diff t,
\end{align}
and we have the following by rearranging terms:
\begin{align}
    \frac{\diff \bar{\vx}(t)}{\diff t} = \frac{\diff \sigma(t)}{\diff t} \epsilon_\theta^{(t)}\left(\frac{\bar{\vx}(t)}{\sqrt{\sigma^2(t) + 1}}\right)
\end{align}
which is equivalent to \Eqref{eq:ddim:differential}. In both cases the initial conditions are $\bar{\vx}(T) \sim \gN(\vzero, \sigma^2(T) \mI)$, so the resulting ODEs are identical.
\end{proof}
\section{Additional Derivations}
\subsection{Accelerated sampling processes}
\label{app:acceleration}

In the accelerated case, we can consider the inference process to be factored as:
\begin{align}
    q_{\sigma, \tau}(\vx_{1:T} | \vx_0) = q_{\sigma, \tau}(\vx_{\tau_{S}} | \vx_0) \prod_{i=1}^{S} q_{\sigma, \tau}(\vx_{\tau_{i-1}} | \vx_{\tau_{i}}, \vx_0) \prod_{t \in \bar{\tau}} q_{\sigma, \tau}(\vx_{t} | \vx_0) \label{eq:diff-new-tau}
\end{align}
where $\tau$ is a sub-sequence of $[1, \ldots, T]$ of length $S$ with $\tau_S = T$, and let $\bar{\tau} := \{1, \ldots, T\} \setminus \tau$ be its complement. Intuitively, the graphical model of $\{\vx_{\tau_i}\}_{i=1}^{S}$ and $\vx_0$ form a chain, whereas the graphical model of $\{\vx_{t}\}_{t \in \bar{\tau}}$ and $\vx_0$ forms a star graph. We define:
\begin{gather}
    q_{\sigma, \tau}(\vx_{t} | \vx_0) = \gN(\sqrt{\alpha_t} \vx_0, (1 - \alpha_t) \mI) \quad \forall t \in \bar{\tau} \cup \{T\} \\
    q_{\sigma, \tau}(\vx_{\tau_{i-1}} | \vx_{\tau_{i}}, \vx_0) = \gN\left(\sqrt{\alpha_{\tau_{i-1}}} \vx_{0} + \sqrt{1 - \alpha_{\tau_{i-1}} - \sigma^2_{\tau_i}} \cdot {\frac{\vx_{\tau_i}  - \sqrt{\alpha_{\tau_i}} \vx_0}{\sqrt{1 - \alpha_{\tau_i}}}}, \sigma_{{\tau_i}}^2 \mI \right) \ \forall i \in [S] \nonumber
\end{gather}
where the coefficients are chosen such that:
\begin{align}
    q_{\sigma, \tau}(\vx_{\tau_i} | \vx_0) = \gN(\sqrt{\alpha_{\tau_i}} \vx_0, (1 - \alpha_{\tau_i}) \mI) \quad \forall i \in [S]
\end{align}
i.e., the ``marginals'' match.

The corresponding ``generative process'' is defined as:
\begin{align}
    p_\theta(\vx_{0:T}) := \underbrace{p_\theta(\vx_T) \prod_{i=1}^{S} p^{(\tau_i)}_\theta(\vx_{\tau_{i-1}} | \vx_{\tau_i})}_{\text{use to produce samples}} \times \underbrace{\prod_{t \in \bar{\tau}} p_\theta^{(t)}(\vx_{0} | \vx_t)}_{\text{in variational objective}} 
\end{align}
where only part of the models are actually being used to produce samples. The conditionals are:
\begin{gather}
    p_\theta^{(\tau_i)}(\vx_{\tau_{i-1}} | \vx_{\tau_i}) = q_{\sigma, \tau}(\vx_{\tau_{i-1}} | \vx_{\tau_i}, f_{\theta}^{(\tau_i)}(\vx_{\tau_{i-1}})) \quad  \text{if} \ i \in [S], i > 1 \\
     p_\theta^{(t)}(\vx_0 | \vx_{t}) = \gN(f_\theta^{(t)}(\vx_t), \sigma_t^2 \mI)  \quad \text{otherwise,}
\end{gather}
where we leverage $q_{\sigma, \tau}(\vx_{\tau_{i-1}} | \vx_{\tau_{i}}, \vx_0)$ as part of the inference process (similar to what we have done in Section~\ref{sec:variational}). The resulting variational objective becomes (define $\vx_{\tau_{L+1}} = \varnothing$ for conciseness):
\begin{align}
   J(\epsilon_\theta) & = \bb{E}_{\vx_{0:T} \sim q_{\sigma, \tau}(\vx_{0:T})}[ \log q_{\sigma, \tau}(\vx_{1:T} | \vx_0) - \log p_\theta(\vx_{0:T})] \\
   & =\bb{E}_{\vx_{0:T} \sim q_{\sigma, \tau}(\vx_{0:T})}\Bigg[ \sum_{t \in \bar{\tau}} \KL(q_{\sigma, \tau}(\vx_{t} | \vx_0) \Vert p_\theta^{(t)}(\vx_0 | \vx_{t}) \\
   & \qquad \qquad \qquad \qquad + \sum_{i=1}^{L} \KL(q_{\sigma, \tau}(\vx_{\tau_{i-1}} | \vx_{\tau_{i}}, \vx_0) \Vert p_\theta^{(\tau_i)}(\vx_{\tau_{i-1}} | \vx_{\tau_i})) ) \Bigg] \nonumber
\end{align}
where each KL divergence is between two Gaussians with variance independent of $\theta$. A similar argument to the proof used in Theorem~\ref{thm:unifiedklobj} can show that the variational objective $J$ can also be converted to an objective of the form $L_\gamma$.




\subsection{Derivation of denoising objectives for DDPMs}
\label{app:ddpm}

We note that in \citet{ho2020denoising}, a diffusion hyperparameter {\color{teal} $\beta_t$}\footnote{In this section we use teal to color notations used in \citet{ho2020denoising}.} is first introduced, and then relevant variables {\color{teal}$\alpha_t := 1 - \beta_t$} and {\color{teal}$\bar{\alpha}_t = \prod_{t=1}^{T} \alpha_t$} are defined. In this paper, we have used the notation $\alpha_{t}$ to represent the variable {\color{teal} $\bar{\alpha}_t$} in \citet{ho2020denoising} for three reasons. First, it makes it more clear that we only need to choose one set of hyperparameters, reducing possible cross-references of the derived variables. Second, it allows us to introduce the generalization as well as the acceleration case easier, because the inference process is no longer motivated by a diffusion. Third, there exists an isomorphism between $\alpha_{1:T}$ and $1, \ldots, T$, which is not the case for {\color{teal} $\beta_t$}.

In this section, we use {\color{teal} $\beta_t$} and {\color{teal} $\alpha_t$} to be more consistent with the derivation in \citet{ho2020denoising}, where
\begin{gather}
{\color{teal} \alpha_t} = \frac{\alpha_t}{\alpha_{t-1}} \\
    {\color{teal} \beta_t} = 1 - \frac{\alpha_t}{\alpha_{t-1}} 
\end{gather}
can be uniquely determined from $\alpha_t$ (i.e. {\color{teal}$\bar{\alpha}_t$}).

First, from the diffusion forward process:
\begin{align}
    q(\vx_{t-1} | \vx_t, \vx_0) = \gN\Bigg(\underbrace{\frac{\sqrt{\alpha_{t-1}}{\color{teal} \beta_t}}{1 - \alpha_t} \vx_0 + \frac{\sqrt{{\color{teal} \alpha_t}} (1 - \alpha_{t-1})}{1 - \alpha_t} \vx_t}_{\color{teal} \tilde{\mu}(\vx_t, \vx_0)}, \frac{1 - \alpha_{t-1}}{1 - \alpha_t} {\color{teal} \beta_t}\mI\Bigg) \nonumber
\end{align}


\citet{ho2020denoising} considered a specific type of $p_\theta^{(t)}(\vx_{t-1} | \vx_t)$:
\begin{align}
    p_\theta^{(t)}(\vx_{t-1} | \vx_t) = \gN\left({\color{teal} \mu_\theta(\vx_t, t)}, \sigma_t \mI\right)
\end{align}
which leads to the following variational objective:
\begin{align}
   {\color{teal} L} & := \bb{E}_{\vx_{0:T} \sim q(\vx_{0:T})} \left[q(\vx_T | \vx_0) + \sum_{t=2}^{T} \log q(\vx_{t-1} | \vx_t, \vx_0) - \sum_{t=1}^{T} \log p_\theta^{(t)}(\vx_{t-1} | \vx_t) \right] \\
   & \equiv \bb{E}_{\vx_{0:T} \sim q(\vx_{0:T})} \left[\sum_{t=2}^{T} \underbrace{\KL(q(\vx_{t-1} | \vx_t, \vx_0)) \Vert p_\theta^{(t)}(\vx_{t-1} | \vx_t))}_{\color{teal} L_{t-1}} - \log p_\theta^{(1)}(\vx_0 | \vx_1) \right] \nonumber
\end{align}
One can write:
\begin{align}
    {\color{teal} L_{t-1}} = \bb{E}_q\left[\frac{1}{2\sigma_t^2} \norm{{\color{teal} \mu_\theta(\vx_t, t)} - {\color{teal} \tilde{\mu}(\vx_t, \vx_0)}}_2^2\right]
\end{align}
\citet{ho2020denoising} chose the parametrization
\begin{align}
    {\color{teal} \mu_\theta(\vx_t, t)} = \frac{1}{\sqrt{{\color{teal} \alpha_t}}} \left(\vx_t - \frac{{\color{teal} \beta_t}}{\sqrt{1 - \alpha_t}} {\color{teal} \epsilon_\theta(\vx_t, t)}\right)
\end{align}
which can be simplified to:
\begin{align}
    {\color{teal} L_{t-1}} = \bb{E}_{\vx_0, \epsilon}\left[\frac{{\color{teal} \beta_t}^2}{2\sigma_t^2 (1 - \alpha_t) {\color{teal} \alpha_t}} \norm{\epsilon - {\color{teal} \epsilon_\theta(\sqrt{\alpha_t} \vx_0 + \sqrt{1 - \alpha_t} \epsilon, t)}}_2^2\right]
\end{align}




\section{Experimental Details}
\label{app:exp}
\subsection{Datasets and architectures}
We consider 4 image datasets with various resolutions: CIFAR10 ($32 \times 32$, unconditional), CelebA ($64 \times 64$), LSUN Bedroom ($256 \times 256$) and LSUN Church ($256 \times 256$). For all datasets, we set the hyperparameters $\alpha$ according to the heuristic in \citep{ho2020denoising} to make the results directly comparable. We use the same model for each dataset, and only compare the performance of different generative processes. For CIFAR10, Bedroom and Church, we obtain the pretrained checkpoints from the original DDPM implementation; for CelebA, we trained our own model using the denoising objective $L_\vone$.

Our architecture for $\epsilon_\theta^{(t)}(\vx_t)$ follows that in \citet{ho2020denoising}, which is a U-Net~\citep{ronneberger2015u} based on a Wide ResNet~\citep{zagoruyko2016wide}. We use the pretrained models from \citet{ho2020denoising} for CIFAR10, Bedroom and Church, and train our own model for the CelebA $64 \times 64$ model (since a pretrained model is not provided). Our CelebA model has five feature map resolutions from $64 \times 64$ to $4 \times 4$, and we use the original CelebA dataset (not CelebA-HQ) using the \href{https://github.com/NVlabs/stylegan/blob/master/dataset_tool.py#L484-L499}{\underline{pre-processing technique}} from the StyleGAN~\citep{karras2018a} repository. %

\begin{table}[H]
    \centering
    \caption{LSUN Bedroom and Church image generation results, measured in FID. For 1000 steps DDPM, the FIDs are 6.36 for Bedroom and 7.89 for Church.}
    \begin{tabular}{c|GGGG|GGGC}
    \toprule
      & \multicolumn{4}{c|}{Bedroom ($256 \times 256$)} & \multicolumn{4}{c}{Church ($256 \times 256$)} \\
    $\dim(\tau)$ & 10 & 20 & 50 & 100 & 10 & 20 & 50 & 100 \\\midrule
    DDIM ($\eta = 0.0$) & \textbf{16.95} & \textbf{8.89} & \textbf{6.75} & \textbf{6.62} & \textbf{19.45} & \textbf{12.47} & \textbf{10.84} & 10.58 \\ %
    DDPM ($\eta = 1.0$) & 42.78 & 22.77 & 10.81 & 6.81 & 51.56 & 23.37 & 11.16 & \textbf{8.27} \\
        \bottomrule
    \end{tabular}
    \label{tab:lsun-fid}
\end{table}

\subsection{Reverse process sub-sequence selection}
\label{app:tau}
We consider two types of selection procedure for $\tau$ given the desired $\dim(\tau) < T$:
\begin{itemize}
    \item \textbf{Linear}: we select the timesteps such that $\tau_i = \floor{c i}$ for some $c$;
    \item \textbf{Quadratic}: we select the timesteps such that $\tau_i = \floor{c i^2}$ for some $c$.
\end{itemize}
The constant value $c$ is selected such that $\tau_{-1}$ is close to $T$. We used \textit{quadratic} for CIFAR10 and \textit{linear} for the remaining datasets. These choices achieve slightly better FID than their alternatives in the respective datasets.

\subsection{Closed form equations for each sampling step}
\label{app:equations}

From the general sampling equation in \eqref{eq:sample-eq-gen}, we have the following update equation:
\begin{align}
    \vx_{\tau_{i-1}}(\eta) = \sqrt{\alpha_{\tau_{i-1}}} \left(\frac{\vx_{\tau_{i}} - \sqrt{1 - \alpha_{\tau_i}} \epsilon_\theta^{(\tau_i)}(\vx_{\tau_{i}})}{\sqrt{\alpha_{\tau_i}}}\right) + \sqrt{1 - \alpha_{\tau_{i-1}} - \sigma_{\tau_i}(\eta)^2} \cdot \epsilon_\theta^{(\tau_i)}(\vx_{\tau_{i}}) + \sigma_{\tau_i}(\eta) \epsilon \nonumber
\end{align}
where
$$\sigma_{\tau_{i}}(\eta) = \eta \sqrt{\frac{1 - \alpha_{\tau_{i-1}}}{1 - \alpha_{\tau_{i}}}}\sqrt{1 - \frac{\alpha_{\tau_i}}{\alpha_{\tau_{i-1}}}}$$
For the case of $\hat{\sigma}$ (DDPM with a larger variance), the update equation becomes:
\begin{align}
    \vx_{\tau_{i-1}} = \sqrt{\alpha_{\tau_{i-1}}} \left(\frac{\vx_{\tau_{i}} - \sqrt{1 - \alpha_{\tau_i}} \epsilon_\theta^{(\tau_i)}(\vx_{\tau_{i}})}{\sqrt{\alpha_{\tau_i}}}\right) + \sqrt{1 - \alpha_{\tau_{i-1}} - \sigma_{\tau_i}(1)^2} \cdot \epsilon_\theta^{(\tau_i)}(\vx_{\tau_{i}}) + \hat{\sigma}_{\tau_{i}} \epsilon \nonumber
\end{align}
which uses a different coefficient for $\epsilon$ compared with the update for $\eta = 1$, but uses the same coefficient for the non-stochastic parts. This update is more stochastic than the update for $\eta = 1$, which explains why it achieves worse performance when $\dim(\tau)$ is small.





\subsection{Samples and Consistency}
\label{app:samples}

We show more samples in \Figref{fig:cifar10-samples} (CIFAR10), \Figref{fig:celeba-samples} (CelebA), \Figref{fig:church-samples} (Church) and consistency results of DDIM in \Figref{fig:celeba-consistency} (CelebA).

\begin{figure}
    \centering
    \begin{subfigure}{\textwidth}
    \centering
    \includegraphics[width=0.8\textwidth]{figures/cifar10-samples-uniform-1000-ddpm.pdf}
    \end{subfigure}
    ~
    \begin{subfigure}{\textwidth}
    \centering
    \includegraphics[width=0.8\textwidth]{figures/cifar10-samples-pt-1000-ddim.pdf}
    \end{subfigure}
    ~
    \begin{subfigure}{\textwidth}
    \centering
    \includegraphics[width=0.8\textwidth]{figures/cifar10-samples-quad-100-ddim.pdf}
    \end{subfigure}
    \caption{CIFAR10 samples from 1000 step DDPM, 1000 step DDIM and 100 step DDIM.}
    \label{fig:cifar10-samples}
\end{figure}

\begin{figure}
    \centering
    \begin{subfigure}{\textwidth}
    \centering
    \includegraphics[width=0.8\textwidth]{figures/celeba-samples-uniform-1000-ddpm.pdf}
    \end{subfigure}
    ~
    \begin{subfigure}{\textwidth}
    \centering
    \includegraphics[width=0.8\textwidth]{figures/celeba-samples-uniform-1000-ddim.pdf}
    \end{subfigure}
    ~
    \begin{subfigure}{\textwidth}
    \centering
    \includegraphics[width=0.8\textwidth]{figures/celeba-samples-uniform-100-ddim.pdf}
    \end{subfigure}
    \caption{CelebA samples from 1000 step DDPM, 1000 step DDIM and 100 step DDIM.}
    \label{fig:celeba-samples}
\end{figure}

\begin{figure}
    \centering
    \includegraphics[width=\textwidth]{figures/celeba-consistency.pdf}
    \caption{CelebA samples from DDIM with the same random $\vx_T$ and different number of steps.}
    \label{fig:celeba-consistency}
\end{figure}

\begin{figure}
    \centering
    \begin{subfigure}{\textwidth}
    \centering
    \includegraphics[width=0.8\textwidth]{figures/church-samples-uniform-100-ddpm.pdf}
    \end{subfigure}
    ~
    \begin{subfigure}{\textwidth}
    \centering
    \includegraphics[width=0.8\textwidth]{figures/church-samples-uniform-100-ddim.pdf}
    \end{subfigure}
    \caption{Church samples from 100 step DDPM and 100 step DDIM.}
    \label{fig:church-samples}
\end{figure}



\subsection{Interpolation}
\label{app:interpolation}
To generate interpolations on a line, we randomly sample two initial $\vx_T$ values from the standard Gaussian, interpolate them with spherical linear interpolation \citep{shoemake1985animating}, and then use the DDIM to obtain $\vx_0$ samples.
\begin{align}
    \vx_T^{(\alpha)} =  \frac{\sin((1 - \alpha) \theta)}{\sin(\theta)} \vx_{T}^{(0)} + \frac{\sin(\alpha \theta)}{\sin(\theta)} \vx_{T}^{(1)}
\end{align}
where $\theta = \arccos\left(\frac{(\vx_T^{(0)})^\top \vx_T^{(1)}}{\norm{\vx_T^{(0)}} \norm{\vx_T^{(1)}}}\right)$. These values are used to produce DDIM samples.

To generate interpolations on a grid, we sample four latent variables and separate them in to two pairs; then we use slerp with the pairs under the same $\alpha$, and use slerp over the interpolated samples across the pairs (under an independently chosen interpolation coefficient). We show more grid interpolation results in \Figref{fig:celeba-interp-grid} (CelebA), \Figref{fig:bedroom-interp-grid} (Bedroom), and \Figref{fig:church-interp-grid} (Church).

\begin{figure}[htbp]
    \centering
    \includegraphics[width=0.6\textwidth]{figures/celeba-interp-grid.png}
    \caption{More interpolations from the CelebA DDIM with $\dim(\tau) = 50$.}
    \label{fig:celeba-interp-grid}
\end{figure}

\begin{figure}[htbp]
    \centering
    \includegraphics[width=0.6\textwidth]{figures/bedroom-interp-grid.png}
    \caption{More interpolations from the Bedroom DDIM with $\dim(\tau) = 50$.}
    \label{fig:bedroom-interp-grid}
\end{figure}

\begin{figure}[htbp]
    \centering
    \includegraphics[width=0.6\textwidth]{figures/church-interp-grid.png}
    \caption{More interpolations from the Church DDIM with $\dim(\tau) = 50$.}
    \label{fig:church-interp-grid}
\end{figure}

\end{document}
