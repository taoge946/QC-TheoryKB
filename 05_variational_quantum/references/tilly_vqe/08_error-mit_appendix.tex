\section{Error mitigation appendix}
\subsection{Common noise models} \label{sec:common_noise_models}
Here we mention a few noise models that may be common in the literature, as for most relevant concepts in quantum computing, readers can refer to Ref.~\cite{nielsenQuantumComputationQuantum2010} for further details.

\subsubsection{Relaxation rates}
\label{sec:mit-noise-relaxation}

We start with the $T_1$ and $T_2$ relaxation time of a quantum computer. They are commonly used as a figure of merit for the noise robustness of a quantum computer. $T_1$ and $T_2$ captures the error on a single qubit quantum state $\rho$ due to thermalization to an equilibrium state with the environment. Specifically, assume the initial quantum state $\rho$ is
\begin{equation}
    \label{eq:error-mit-noise-model-rho}
    \rho = \begin{pmatrix}
        a     & b   \\
        b^{*} & 1-a
    \end{pmatrix}.
\end{equation}
Then, after a period of time $t$, the quantum state is changed by the noise $\mathcal{N}$ into~\cite{nielsenQuantumComputationQuantum2010}

\begin{equation}
    \mathcal{N}(\rho) = \begin{pmatrix}
        ( a-a_{0}) e^{-t/T_{1}} +a_{0} & be^{-t/T_{2}}                     \\
        b^{*} e^{-t/T_{2}}             & ( a_{0} -a) e^{-t/T_{1}} +1-a_{0}
    \end{pmatrix},
\end{equation}
where $a_{0}$ represents the thermal equilibrium state, and usually $a_{0}=1$ on superconducting qubits.
The parameters $T_{1}$ and $T_2$ are also known as spin–lattice (or `longitudinal') and spin–spin (or `transverse') relaxation rates. They can be obtained experimentally are widely available metrics for most quantum computers today.

\subsubsection{Over-rotation}

This type of error happens when a physical pulse used to generate a particular gate are mis-calibrated
\cite{merrillProgressCompensatingPulse2012}. In the case or Pauli rotation gates $R_{i}(\theta)$, $i\in\{x,y,z\}$,
this error manifests a small deviation $\delta$ in the angle $\theta$, resulting in an actual gate of modified
rotation gate $R_i(\theta + \delta)$. The deviation may be systematic\cite{bultriniSimpleMitigationStrategy2020}, but may also be stochastic and usually varies differently from different quantum computer.


\subsubsection{Depolarizing noise model}

The depolarizing model $\mathcal{N}_{\text{depolar}}$ is a common noise model. On a quantum state $\rho$ of $n$ qubits, it changes $\rho$ according to

\begin{equation}
    \mathcal{N}_{\text{depolar}}( \rho ) =( 1-\varepsilon ) \rho +\varepsilon \frac{\unit{}}{2^{n}},
\end{equation}
where $\unit{}$ is the identity matrix on $n$ qubits, and $\varepsilon$ characterizes the strength of the depolarizing noise.

\subsubsection{Dephasing noise model}

The dephasing noise model $\mathcal{N}_{\mathrm{dephasing}}$ is related to the disappearance of off-diagonal part of a density matrix, hence closely related to the $T_2$ relaxation rate of a quantum computer. Its action on a single qubit state $\rho$ is described by

\begin{equation}
    \begin{aligned}
        \mathcal{N}_{\mathrm{dephasing}}(\rho ) & =( 1-\varepsilon ) \rho +\varepsilon Z\rho Z \\
                                                & =\left(\begin{array}{ c c }
                a                       & b( 1-2\varepsilon ) \\
                b^{*}( 1-2\varepsilon ) & 1-a
            \end{array}\right),
    \end{aligned}
\end{equation}
where $a$ and $b$ characterize the initial state of $\rho$ (Eq.~\ref{eq:error-mit-noise-model-rho}), $\varepsilon$ characterizes the strength of this noise model. It's clear that the depolarizing noise does not affect the diagonal part of a single qubit state, therefore it commutes with most symmetry operator and could not be detected by the symmetry verification technique mentioned in Sec.~\ref{sec:mit-symmetry-verification}.

\subsubsection{Damping error}

Here as suggested by the name, the damping error decreases the population of $\ket{1}$ in a single qubit state, and is closely related to the $T_1$ relaxation time.
This error model can be described by the Kraus operators
\begin{equation}
    K_{0} =\begin{pmatrix}
        1 & 0                     \\
        0 & \sqrt{1-\varepsilon }
    \end{pmatrix},\;
    K_{1} =\begin{pmatrix}
        0 & 0                   \\
        0 & \sqrt{\varepsilon }
    \end{pmatrix},
\end{equation}
where $\varepsilon$ characterizes the strength of this noise model. Its action on the initial state $\rho$ in Eq.~(\ref{eq:error-mit-noise-model-rho}) is

\begin{equation}
    \begin{split}
        & \mathcal{N}_{\mathrm{damping}}(\rho) = \sum_{i=0}^1 K_i \rho K_i^\dagger \\
        & = \begin{pmatrix}
            1+(a-1)(1-\varepsilon )    & b\sqrt{1-\varepsilon } \\
            b^{*}\sqrt{1-\varepsilon } & (1-a)(1-\varepsilon )
        \end{pmatrix}.
    \end{split}
\end{equation}

\subsection{Example for probabilistic error cancellation}
\label{sec:mit-pec-toy-example}

Here we present an illustrative example of using probabilistic error cancellation to correct the depolarizing error on a one-qubit quantum computer. The depolarizing noise $\mathcal{N}_{\mathrm{dep}}$ changes an initial state $\rho$ to

\begin{equation}
    \label{eq:mit-pec-2}
    \mathcal{N}_{\mathrm{dep}}( \rho ) =\left( 1-\frac{3}{4} p\right) \rho +\frac{p}{4}\sum _{\sigma \in \{X,Y,Z\}} \sigma \rho \sigma .
\end{equation}
Here $p\in \left[ 0,\frac{4}{3}\right]$ characterizes the strength of the depolarizing noise, and $\{X,Y,Z\}$ are the Pauli matrices. One can show that $\mathcal{N}^{-1}_{\mathrm{dep}}$ can be written as~\cite{temmeErrorMitigationShortDepth2017}

\begin{equation}
    \label{eq:mit-pec-3}
    \begin{split}
        \mathcal{N}^{-1}_{\mathrm{dep}}( \rho ) & =\gamma _{\mathrm{PEC} ,\ \mathrm{dep}}\left( q_{\mathrm{dep} ,1} \rho +q_{\mathrm{dep} ,2}\sum _{\sigma \in \{X,Y,Z\}} \sigma \rho \sigma \right) ,\\
        \gamma _{\mathrm{PEC} ,\ \mathrm{dep}} & =\frac{p+2}{2-2p} ,\\
        q_{\mathrm{dep} ,1} & =\frac{4-p}{2p+4} ,\\
        q_{\mathrm{dep} ,2} & =-\frac{p}{2p+4} .
    \end{split}
\end{equation}

Suppose the effective quantum channel of an ideal unitary $U$ on quantum computer could be characterized as $\mathcal{N}_{1} \circ U$, where the circle $\circ $means composition of quantum channels. One could define three additional quantum circuits, each formed by appending one of the three Pauli gates after the unitary $U$.
One obtains the measurement outcome from the four circuits $\{U,X\circ U,Y\circ U,Z\circ U\}$ containing the original one, and calculates according to the weights in Eq.~(\ref{eq:mit-pec-3})

\begin{equation}
    \begin{split}
        \label{eq:mit-pec-4}
        & \langle \hat{O} \rangle_{\mathrm{PEC} ,\ \mathrm{dep}}                                                                                                                                                \\
        = & \gamma _{\mathrm{PEC} ,\ \mathrm{dep}}\bigg(q_{\mathrm{dep} ,1}\underbrace{\operatorname{Tr}[\hat{O} \mathcal{N}_{1} \circ U( \rho )]}_{\text{original circuit}} + q_{\mathrm{dep} ,2}\sum _{\sigma \in \{X,Y,Z\}}\underbrace{\operatorname{Tr}[ \hat{O} \sigma \circ \mathcal{N}_{1} \circ U( \rho )]}_{\text{appended by a Pauli gate}}\bigg)               \\
        = & \operatorname{Tr}\Big[\hat{O}\underbrace{\mathcal{N}^{-1}_{1}(\mathcal{N}_{1} \circ U( \rho ))}_{\text{quasi-probability\ representation\ of\ } U\ \text{acting\ on} \ \rho }\Big] \\
        = & \operatorname{Tr}[ \hat{O} U( \rho ))].
    \end{split}
\end{equation}
Therefore, the noise effect of $\mathcal{N}_{1}$ can be canceled with the PEC method.

\subsection{Implementation of exponential error suppression}
\label{sec:derangement-details}

%the first method
Here we will mention several methods to actually compute the value in Eq.~(\ref{eq:ees-computed-rhon}) using quantum computers. For simplicity, we mostly consider the computation of the numerator
\begin{equation}
    \label{eq:ees-numerator}
    \mathrm{Tr( \rho^{\nEES} \hat{O})},
\end{equation}
since the denominator is a special case of the numerator with $\hat{O}$ being the identity observable.

\subsubsection{Ancilla assisted method}
\label{sec:derangement-details_ancilla-assisted}
The first method needs the assistance of an extra ancilla qubits. Considering
$n$ copies of the same state $\rho^{\otimes \nEES}$. We want to measure the observable $O$ only on one system, but
in as an overlap of the original state, and a permuted state. Specifically, we want to compute the quantity

\begin{equation}
    \label{eq:ees-single1}
    \begin{aligned}
          & \langle \psi _{i_1} |\otimes \cdots \otimes \langle \psi _{i_n} |O^{( 1)} |\psi _{s ( i_{1})} \rangle \otimes \cdots \otimes |\psi _{s ( i_{\nEES})} \rangle \\
        = & \langle \psi _{i_{1}} |O|\psi _{i_{1}} \rangle \delta _{i_{1} ,s ( i_{1})} \cdots \delta _{i_{n} ,s ( i_{\nEES})},
    \end{aligned}
\end{equation}
where for simplicity we consider only one possibility
$|\psi _{i_1} \rangle\otimes \cdots \otimes |\psi _{i_\nEES}\rangle$ in the
statistical mixture $\rho^{\otimes \nEES}$,
$O^{(1)}$ is the same observable $O$ on any one of the $n$ copies of the same system,
$s$ is called the $\nEES$-cycle derangement in the Group theory. Explicitly, $s$ is a permutation in the indices $\{1,2,\cdots \nEES\}$, and can be written as mappings
\begin{equation}
    \label{eq:ees-VD-ncycle}
    \begin{aligned}
        s( i_{1})   & =i_{2}\\
        s( i_{2})   & =i_{3}   \\
                    & \cdots   \\
        s( i_{n-1}) & =i_{\nEES}   \\
        s( i_{n})   & =i_{1} ,
    \end{aligned}
\end{equation}
where all indices $\{i_1,\cdots,i_\nEES\}$ are distinct numbers. An example is
\begin{equation}
    \label{eq:ees-cycle-all}
    s(i) = i+1,\,(i< \nEES),\, s(\nEES) = 1.
\end{equation}

$s$ being a permutation of all indices (i.e. $\{i_1,\cdots,i_\nEES\}$ are distinct numbers above)
enforces the delta functions $\delta _{i_{1} ,s ( i_{1})} \cdots \delta _{i_{n} ,s ( i_{\nEES})}$
in Eq.~(\ref{eq:ees-single1}).
For the statistical mixture $\rho^{\otimes \nEES}$, we are effectively computing $\operatorname{Tr}[\rho^\nEES O]$ since

\begin{equation}
    \begin{aligned}
          & \sum _{i_{1} \cdots i_{n}} p_{i_{1}} \cdots p_{i_{\nEES}}                                                                                                                         \\
          & \times \langle \psi _{i_{1}} |\otimes \cdots \otimes \langle \psi _{i_{\nEES}} |O^{( 1)} |\psi _{\sigma ( i_{1})} \rangle \otimes \cdots \otimes |\psi _{\sigma ( i_{\nEES})} \rangle \\
        = & \sum _{i_{1}} p_{i_{1}}^{\nEES} \langle \psi _{i_{1}} |O|\psi _{i_{1}} \rangle =\operatorname{Tr}\left[ \rho ^{\nEES} O\right].
    \end{aligned}
\end{equation}

Given an explicit expression for $s$, it is easy to construct a circuit $D_\nEES$ which performs the unitary mapping
\begin{equation}
    |\psi _{i_1} \rangle\otimes \cdots \otimes |\psi _{i_\nEES}\rangle
    \mapsto
    |\psi _{s ( i_{1})} \rangle \otimes \cdots \otimes |\psi _{s ( i_{\nEES})} \rangle.
\end{equation}
For example, for the $s$ in Eq.~(\ref{eq:ees-cycle-all}), $D_m$ can be constructed by
performing CNOT operations on pairs of qubits $(1,2),\cdots,(\nEES-1,\nEES),(\nEES,1)$.
Having constructed $|\psi _{s ( i_{1})} \rangle \otimes \cdots \otimes |\psi _{s ( i_{\nEES})} \rangle$,
measuring the overlap in Eq.~(\ref{eq:ees-single1}) could be achieved easily with a modified
Hadamard test, which is illustrated in the circuit diagram in Fig.~\ref{fig:ees-general-circ}.

We note that several improvements of the derangement operation $D_\nEES$ exists in the literature~\cite{hugginsVirtualDistillationQuantum2021, czarnikQubitefficientExponentialSuppression2021}. In particular,
\citet{czarnikQubitefficientExponentialSuppression2021} utilize a deep circuit to achieve derangement on a single copy, avoiding creating $\nEES$ copies of $\rho$ and applying the derangement operation on all $\nEES$ copies.
Although $D_\nEES$ seems to be prone to errors on actual quantum computer due to its long-range nature,
as long as the error does not affect the orthogonal relations in Eq.~(\ref{eq:ees-single1}), the noisy $D_\nEES$
is still a valid derangement operation, and it is shown in \citet{koczorExponentialErrorSuppression2021}
that the error induced by $D_\nEES$ could be
mitigated effectively with extrapolation based mitigation techniques.


\subsubsection{Diagonalization method}
\label{sec:derangement-details_diagonalization}
In the second method, we reformulate the derangement operation
as a matrix $S$, and implement a unitary version of
$S$ and the observable after the derangement operation $OS$ in order to
estimate the denominator and the numerator of Eq.~(\ref{eq:ees-computed-rhon}).
Compared with the previous method, this method does not require additional
ancilla qubit and the long range controlled-$D_\nEES$ operation (Fig.~\ref{fig:ees-general-circ}). This method only performs local operations $B_i$ which connects the same set of qubits in each copy of the state $\rho$ in $\rho^{\otimes \nEES}$.
We will first describe a general method, and then specialize on a specific case ($\nEES=2$, $O$ acts only one qubit only)
to illustrate the potential of this method.

We can define a matrix $S$ which swaps quantum systems as defined by the derangement permutation
$s$ in Eq.~(\ref{eq:ees-VD-ncycle})
\begin{equation}
    \label{eq:ees-derangement-matrix-S}
    S | \psi _{i_1} \rangle \otimes \cdots \otimes | \psi _{i_\nEES} \rangle
    = |\psi _{s ( i_{1})} \rangle \otimes \cdots \otimes |\psi _{s ( i_{\nEES})} \rangle.
\end{equation}
It should be easy to verify the equalities (see for example Eq.~\ref{eq:ees-single1})
\begin{equation}
    \label{eq:ees-VD-trOSrho}
    \mathrm{Tr}(O^{(1)} S \rho^{\otimes \nEES}) = \mathrm{Tr}(O \rho^\nEES),\,
    \mathrm{Tr}(S \rho^{\otimes \nEES}) = \mathrm{Tr}(\rho^\nEES), 
\end{equation}
where the observable $O^{(1)}$ is the same observable $O$ on the first one of the $n$ copies of the system.
It should be noted that $S$ in Eq.~(\ref{eq:ees-VD-trOSrho}) is not intended to define a unitary operation,
since it acts on the density matrix $\rho$ as $S\rho$ instead of $S\rho S^\dagger$.
Although $S$ and $O^{(1)}S$ are not Hermitian in general, in many cases they may be diagonalized by unitary matrices.
That is, we could find unitary matrices $B_S$ and $B_{OS}$ such that
\begin{equation}
    \label{eq:ees-sym-diag-gates}
    B_S \Lambda_S B_S^\dagger = S,\, B_{OS} \Lambda_{OS} B_{OS}^\dagger = O^{(1)} S,
\end{equation}
where $\Lambda_S$ and $\Lambda_{OS}$ are diagonal matrices.
Then, in principle we may perform the unitaries $B_S$ and $B_{OS}$ on the state $\rho^{\otimes \nEES}$,
and measure the $\nEES$ copies of $\rho$ each in their computation basis to obtain the numerator and denominator of
Eq.~(\ref{eq:ees-computed-rhon}). Naively, it may seem difficult to calculate the diagonalization since both $S$ and $O^{(1)}S$ operate on the large Hilbert space of $\rho^{\otimes \nEES}$. However, since $S$ permutes the $\nEES$ copies of $\rho$,
$S$ can be easily factorized into tensor products
\begin{equation}
    S = S_1 \otimes S_2 \otimes \cdots \otimes S_N,
\end{equation}
where $N$ is the number of qubits of state $\rho$, $S_{i}$ permutes all the $i$-th qubit on each copy.
For example, when $\nEES=2$, $S_i$ interchanges the qubits in the same position on the two copies of $\rho$,
i.e.
\begin{equation}
    S_i = \begin{pmatrix}
        1 & 0 & 0 & 0 \\ 0& 0& 1& 0\\ 0& 1& 0& 0\\ 0& 0& 0& 1
    \end{pmatrix}.
\end{equation}

Therefore, for the qubits which are not related to the observable $O$, diagonalizing $S$ is simplified
into diagonalizing $S_i$, which would not be difficult assuming $\nEES$ is not large.
Meanwhile, in many cases of VQE, the observable $O$ may be limited to act only on a few qubits.
In this case, the diagonalization of $O^{(1)}S$ could be simplified into the diagonalization of $S_i$
above, and the diagonalization of
\begin{align}
    O \otimes_{i\in \mathbb{S}_O} S_i,
\end{align}
where $\mathbb{S}_O$ is the subset of qubits on which $O$ acts.
Furthermore, when $O$ is a one-qubit observable (e.g. $O=Z_1$), we can define the symmetrized
version of $O$ as
\begin{equation}
    O_{\mathrm{sym}} = \sum_{j=1}^\nEES O^{(j)}.
\end{equation}
where $O^{(j)}$ is the same observable $O$ on the $j$-th copy among $n$ copies of $\rho$.
Obviously $O_{\mathrm{sym}}$ and $S$ commutes. Additionally, $O_{\mathrm{sym}}$ and $S$
can be factorized into local unitaries acting on the same qubit in the $\rho$ among the $\nEES$ copies.
Hence, the two diagonalization unitaries coincide ($B_O = B_{OS}$) and can also be factorized
as tensor products of unitaries acting only on the same qubit in the $\rho$ among the $\nEES$ copies.
In particular,
we illustrate an explicit example when $\nEES=2$ of this method in Fig.~\ref{fig:ees-sym-diag}. Sec.~[II.A] of \citet{hugginsVirtualDistillationQuantum2021} also gives the explicit formula for $\Lambda_{O/OS}$ and diagonalizing
unitary $B_{O/OS}$ when $\nEES=2$.

Finally, using grouping methods (see Sec. \ref{sec:Grouping}), we can turn a complex multiple qubit measurement into a simple one-qubit measurement, making this implementation more appealing since it only needs
local qubit connections.

\begin{figure}[ht]
    \centering
    \includegraphics[width=0.50\linewidth]{figs/errormit/ees-sym-diag.pdf}
    \caption{An illustration of the exponential error suppression mitigation implemented with the simultaneous diagonalization method mentioned in Sec.~\ref{sec:derangement-details}
    (adopted from \citet{hugginsVirtualDistillationQuantum2021}).
    We prepare two copies of the same quantum state using the same ansatz circuit $U(\boldsymbol{\theta})$,
    and then apply the diagonalizing gate on each pair of the qubit as specified by Eq.~(\ref{eq:ees-sym-diag-gates}),
    and measure the observables in the computation basis (Eq.~\ref{eq:ees-sym-diag-gates}) in the computation basis.
}
    \label{fig:ees-sym-diag}
\end{figure}

\subsubsection{Dual state purification method}
\label{sec:derangement-details_dual-state-pure}

\begin{figure}[ht]
    \centering
    \subfloat[]{
        \label{fig:dual-state-purification-a}
        \includegraphics[width=0.5\linewidth]{figs/errormit/dual-state-purification-a.pdf}
    }
    \subfloat[]{
        \label{fig:dual-state-purification-b}
        \includegraphics[width=0.40\linewidth]{figs/errormit/dual-state-purification-b.pdf}
    }
    \caption{Quantum circuit to measure Pauli $Z$ on the first qubit using the dual state purification method in exponential error supprresion (Appendix~\ref{sec:derangement-details_dual-state-pure}). 
    In \protect\subref{fig:dual-state-purification-a}, a mid-circuit measurement labelled by $\mathrm{mid}$ is carried out in addition to the final
    measurement on all qubits, and only experiments where all the final qubits are measured to be $0$ are considered.
    In \protect\subref{fig:dual-state-purification-b}, no mid-circuit
    measurement is carried out, and it should be noted that dual to the noise present in the quantum computer, the combined effect of applying $U$ and $U^\dagger$ is not identity in \protect\subref{fig:dual-state-purification-b}.
    }
    \label{fig:dual-state-purification}
\end{figure}

It was shown in \citet{huoDualstatePurificationPractical2021} that for the special case $\nEES=2$, exponential error suppression could be achieved without creating multiple copies of the state $\rho$, and without any controlled unitary gates. The disadvantage is that this method requires measuring all qubits as a final step, and it doubles the circuit depth. The core idea is to ``measure'' $\rho$ on the state $O\rho$, therefore giving the observable $\operatorname{Tr}(\rho O\rho) = \operatorname{Tr}(O\rho^2)$.
It should be noted that, although that same spirit can be applied to the general case where $m$ is an even number $\nEES=2, 4, etc$~\cite{cai2021resourceefficient}, only in the case of $\nEES=2$ can we achieve exponential error suppression without the over-head of controlled unitary gates.

The circuit diagram in Fig.~\ref{fig:dual-state-purification} shows an example of dual state purification method when the observable to be measured is Pauli-$Z$ on the first qubit (denoted by $Z_1$).
In the diagram, the circuit $U$ which prepares a state $\rho$ from $\ket{0}$ is applied twice:
$U$ in the first time and its inverse $U^\dagger$ in the second time.
In the first diagram (Fig.~\ref{fig:dual-state-purification-a}),
a mid-circuit measurement is carried out, and one also post-select on the experiments where the final measurement on all qubits gives the same outcome $0$. In this case, one denotes the probability that the final measurement obtaining $0$ in all qubits when the mid-circuit measurement gives an outcome of $b\in{0,1}$ as $P_{\mathbf{0}, b}$.
Also, in a separate experiment illustrated in Fig.~\ref{fig:dual-state-purification-b}, the mid-circuit measurement $\nEES$ is not carried out, and one
denotes the probability that the final measurement obtaining $0$ in all qubits as $P_{\mathbf{0}}$.
Then, the two circuits outcome can be combined to the following value, which closely approximate $\mathrm{Tr}(Z_1 \rho^2)/\mathrm{Tr}(\rho^2)$:
\begin{equation}
    \frac{P_{\mathbf{0}, 0} - P_{\mathbf{0}, 1}}{P_{\mathbf{0}}}.
\end{equation} 
This implementation can be adapted to measure any quantum observable $O$ by expanding $O$ into a linear sum of Pauli observables. Details could be found in \citet{huoDualstatePurificationPractical2021} 
and an explicit extension of the circuit in Fig.~\ref{fig:dual-state-purification} to arbitrary observable $O$ which squares to
the identity (e.g. Pauli observables) is available in the Appendix A of~\citet{cai2021resourceefficient}.

\subsubsection{Shadow tomography based method}
\label{sec:derangement-details_shadow-distillation}

This method has been proposed in \citet{Seif2022ShadowDistillation,Hu2022LogicalShadowTomography}, and uses the shadow tomography technique (see Sec.~\ref{sec:pauli_grouping-inference-methods}) to estimate the quantity
in Eq.~(\ref{eq:ees-computed-rhon}), or more simply the numerator  $\mathrm{Tr( \rho^{\nEES} \hat{O})}$ (Eq.~\ref{eq:ees-numerator}).
The method proposed in \citet{Hu2022LogicalShadowTomography} is more complicated and involves error correction codes, but shares the same spirit with \citet{Seif2022ShadowDistillation} in using shadow tomography to estimate the numerator $\mathrm{Tr( \rho^{\nEES} \hat{O})}$.
By applying a random unitary $U$ generated from a pool on $\rho$, and measuring it on the computational basis, we can obtain a classical shadow $\tilde{\rho}$ of $\rho$ (see Sec~\ref{sec:pauli_grouping-inference-methods} for details), and compute the numerator using the shadow $\tilde{\rho}$ as a surrogate for $\rho$.
An explicit formula is available in \citet{Seif2022ShadowDistillation} when the pool of random unitaries satisfy the 3-design property. 
One significant advantage of the shadow tomography based method over other methods presented above, is that shadow tomography does not require additional copies of the same state (Sec.~\ref{sec:derangement-details_ancilla-assisted} and \ref{sec:derangement-details_diagonalization})
or doubling of the depth of the circuit (Sec.~\ref{sec:derangement-details_dual-state-pure}).
On the other hand, the number of required random unitaries ($N_U$) to sample in order to
achieve an estimate of $\mathrm{Tr( \rho^{\nEES} \hat{O})}$ within certain precision, 
might scale exponentially with respect to the number of qubits $N$. A preliminary numerically study in \citet{Seif2022ShadowDistillation} suggest that $N_U\approx 2^{0.82N}$ for $\nEES=2$, which is better than the cost of performing state tomography on $\rho$,
but is out of reach for large scale quantum experiments.